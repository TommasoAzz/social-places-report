\documentclass[../../main]{subfiles}
\begin{document}

\subsection{Application implementation choices}
\label{ss:application-implementation-choices}

In \hyperref[ss:application-architecture]{Section 2.2} we expressed how the application is designed from a high-level perspective and how the different modules of the application should communicate between each other.\\
We shall now present the choices we took in terms of technologies and implementation of the actual modules.

\subsubsection{The operating system}
The operating system we chose to develop the app for is \textit{Android} because it is more widespread and the resources available on the Internet are much more, compared for example to those available for \textit{iOS}.
The tools we used for the development of the app are the \textbf{Android SDK} and \textbf{Android Studio} as an IDE, very helpful for developing Android applications and very well integrated within the Android ecosystem.

\subsubsection{The authentication means}
Instead of designing an authentication system ourselves, which would have taken a lot of time and resources (and possibly, bad security), we decided to opt for Google's \textbf{Firebase Authentication}, given its ease of use and availability for both the \textbf{Android SDK} and the technologies chosen for developing the backend, which are presented in \hyperref[ss:server-implementation-choices]{Section 3.2}.
Since we wanted to focus more on the actual functionalities of the application, we took advantage of the Android's authentication UI for asking the user to authenticate.
This interface, in a scenario in which the user actually logs in, returns a specific type of credentials which can be handled by the \textbf{Firebase Authentication} Android library for logging the user in and returning an API token, to be attached to every API call to the backend.

\subsubsection{The push notification means}
Since we were already using a \textbf{Firebase} product, we decided to use \textbf{Firebase Cloud Messaging} by Google, too, for managing the push notifications, since it also free to use.
Its configuration is actually pretty straightforward and only requires to specify which possible push notifications can the application receive and what to do when the application updates the so-called \textit{notification token}.
In fact, when the application is at its first run or when the \textbf{Firebase Cloud Messaging} Android library believes it is necessary, a \textit{notification token} is sent to the backend of the system through an API call.
This is used by the backend to target specific devices (i.e., users) when notifications need to be dispatched.
Every time a notification is received, the application handles it and on a tap by the user on it, the UI gets updated accordingly.

\subsubsection{The UI}
We kept the User Interface as essential as possible since we have no specific graphical skills in designing them.\\
For the Android platform there are currently two ways for implementing the user interface:
\begin{itemize}
    \item \text{XML Layout}, which is the default way to create user interfaces, declaring an XML tree like one would for a web page, using tags from the \textit{Android XML Schema} and controlling its behaviour inside the Java/Kotlin codebase;
    \item \text{Jetpack Compose}, which is a new way for building user interfaces in Android, that does not require any XML.
\end{itemize}
Since when we started the project \textbf{Jetpack Compose} was not stable yet for the public use, we decided to stick with \textbf{XML Layout}.

\subsubsection{The geofence transitions manager and periodically activated tasks manager}
In order to receive \textit{location-aware} recommendations (i.e., those based on the user entrance in a \textit{geofence}) and \textit{periodical} recommendations we implemented the services the \textbf{Android SDK} provides for these purposes.\\
In detail:
\begin{itemize}
    \item for the \textit{location-aware} recommendations, we used Android's \textbf{Geofencing Client}, in which we can register a \textit{geofence} for each \textit{poi} the user has in its \textit{pois} list and every time a new one is created.
    Every \textit{geofence} is a circular area with centre in the \textit{poi} coordinates and radius to be programmatically set (we chose 300 metres);
    \item for the \textit{periodical} recommendations, we used Android's \textbf{Alarm Manager}, in which we can register tasks to be executed every time a previously set alarm fires.
\end{itemize}

\subsubsection{The location updates manager and human activity recognition}
Getting location updates and the current human activity is made possible again by implementing the services of the \textbf{Android SDK} for these purposes.\\
In detail:
\begin{itemize}
    \item the current location is obtained using Android's \textbf{Fused Location Provider Client}, in which we can set the precision of the returned coordinates and the update interval;
    \item the current human activity is instead obtained using Android's \textbf{Activity Recognition Client}, with which we can get a list of possible current human activities with their level of accuracy.
\end{itemize}

\subsubsection{The API connectors and business logic}
Connecting to the backend through its API requires an HTTP client since the server implements RESTful APIs, as it is explained more in detail in \hyperref[ss:server-implementation-choices]{Section 3.2}.
For this reason we chose the \textbf{Retrofit} library which allows the developer to focus mainly on the data aspect of the communication, that is, the input and output arguments, and of course which endpoints to contact.\\
By the way, some work still need to be done: error handling, for example. That is why there is also a module of the application, the \textbf{business logic}, which purpose is to create an extra layer between the bare API calls and the UI.


\end{document}