\documentclass[../../main]{subfiles}
\begin{document}

\label{ss:description-project-idea}
\subsection{Description of the project idea}
Very often it happens that you do not know what to do even knowing many different kinds of places where you could go.
This is where the idea is born, a social recommendation app where a user can save his favourite places and share them with his friends.
Starting from this thinking we choosed \textbf{Social Places} as name for our application.
The user can interact with three main subject:
\begin{itemize}
    \item Point Of Interest;
    \item Live Events;
    \item Friends;
\end{itemize}
A user can create and share a \textbf{point of interest} (from here we'll refere to it as \textbf{poi}), or create a \textbf{live event} that will be notified to all his friends, or interact with other people by sending, confirming, denying a friendship request or removing a friend.
But the most interesting part is when the app interact with the user by recommending a suitable place for the actual user situation.

There are two type of recommendation:
\begin{itemize}
    \item Periodic Recommendation, based on a timer;
    \item Geofence Recommendation, based on the area of a poi;
\end{itemize}
As input for the recommendation model we provide:
\begin{itemize}
    \item Time of day (seconds);
    \item Day of week (number);
    \item Human activity recognized (activity name);
    \item Actual position (latitude, longitude).
    \item And the category of the place, in case of a Geofence Recommendation (place type).
\end{itemize}
\textbf{IN UNALTRA SEZIONE QUELLO SOTTO}

    \paragraph*{Points Of Interest}
A Point Of Interest is a collection of information describing a place, represented by a marker, on the map. 
It is composed of:
\begin{itemize}
    \item \textbf{Name:} A personal name describing the place;
    \item \textbf{Address:} The address of the marker on the map, caputered via google maps API;
    \item \textbf{Phone number:} Telephone number of the place, if found via google maps API;
    \item \textbf{Url:} Website url of the place, if found via google maps API;
    \item \textbf{Type:} The category of the place:
        \begin{itemize}
            \item \textbf{Restaurant;}
            \item \textbf{Sport;}
            \item \textbf{Leisure;}
            \item \textbf{Live Event;} 
        \end{itemize}
    \item \textbf{Visibility:} The Visibility of the place. If it is public the user's friends can see it, if it is private only the owner can see it;
    \item \textbf{Latitude:} Latitude of the place;
    \item \textbf{Longitude:} Longitude of the place;
\end{itemize}
    


\end{document}