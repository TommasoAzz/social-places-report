\documentclass[../../main]{subfiles}
\begin{document}

\subsection{Security in the communication}
\label{ss:security-in-communication}

\subsubsection{HTTPS Server}
\label{sss:https-server}
A security mechanism, to avoid the communications being intercepted, needs to be implemented.\\
One initial idea is to use HTTPS as a communication protocol between the client (mobile application) and the server (the APIs), but we believe this is not enough.
We want to provide a way to obscure sensitive information during the communication, which is the location of the user that needs to be sent in order to receive recommendations.

\subsubsection{Obscured communication}
\label{sss:obscured-communication}
There are many techniques to hide sensitive user data. In our scenario, that would be only location data and for that there exists such like spatial cloaking or dummy updates.
Since we want to know the actual current position of the user, we decided that it was best to leave the location data as-is and instead use an encrypted end-to-end communication, exploiting the RSA scheme.
Not all communications are encrypted in the final product, only those that comprehend sensitive data, i.e., the recommendations.
Those that are left are anyway still protected by the HTTPS communication.

\end{document}